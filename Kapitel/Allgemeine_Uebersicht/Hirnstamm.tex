\subsection{Hirnstamm}
\label{subsec:Hirnstamm} \index{Hirnstamm}
%%%%%%%%%%%%%%%%%%%%%%%%%%%%%%%%%%%%%%%%%%%%%%%%%%%%%%%%%%%
%%%%%%%%%%%%%%%%%%%%%%%%%%%%%%%%%%%%%%%%%%%%%%%%%%%%%%%%%%%

Unter dem Begriff Hirnstamm (\textit{Truncus cerebri}) werden die ventralen Bereiche des Mittelhirns (Mesencephalon), des Hinterhirns (Metencephalon) und des Nachhirns (Myelencephalon) zusammengefasst. Es besteht somit aus Teilen des Mittelhirns, sowie aus dem Pons und der Medulla. Rostral, vor dem Hirnstamm, liegt das Zwischenhirn (Diencephalon), caudal, hinter dem Hirnstamm, folgt das Rückenmark. Im Hirnstamm sind motorische Zentren lokalisiert, die für die Kontrolle der Körperhaltung und Bewegung zuständig sind. Diese Bereiche werden im Allgemeinen auch als \textbf{Tegmentum} bezeichnet. Es kann, dem Verlauf des Ventrikelsystems folgend, in drei Unterbereiche gegliedert werden: Das \textit{Tegmentum mesencephali}\index{Tegmentum! mesencephali} befindet sich im ventralen, bzw. inferioren Mittelhirn \textsuperscript{\cite[Kap.~6]{trepel2011neuroanatomie}}. Das \textit{Tegmentum pontine}\index{Tegmentum! mesencephali} ist ebenfalls ventral gelegen und befindet sich im Pons. Der Bereich des Tegmentums, der sich in der Medulla befindet, wird \textit{Tegmentum myelencephali}\index{Tegmentum! myelencephali} genannt. Im Vergleich zum Mes- und Metencephalon ist das Tegmentum im Myelencephalon dorsal, bzw. superior gelegen \textsuperscript{\cite[Kap.~5]{trepel2011neuroanatomie}}.\\

\noindent Neben den motorischen Funktionen besitzt der Hirnstamm auch andere Aufgaben. So liegen in ihm auch relativ autonome, vegetative Zentren, die beispielsweise Atem- und Kreislaufsystem steuern \textsuperscript{\cite[Kap.~14]{penzlin2005tierphys}}. Einige dieser Zentren befinden sich in der \textbf{Formatio reticularis}\index{Formatio reticularis}. Sie erstreckt sich vom Mesencephalon über den Pons und die Medulla bis hinab ins Rückenmark. Durch die Formatio reticularis werden sowohl sympathische als auch parasympathische Zentren gesteuert. Sie selbst wird vom Hypothalamus kontrolliert. Funktionell lässt sich die Formatio reticularis in eine Vielzahl kleinerer Zentren unterteilen. Sowohl motorische Zentren, wie das sogenannte motorische Zentrum und das Augenbewegungszentrum (Kap.~\ref{sec:Motorik}), als auch Kreislauf- und Atemzentren sind in ihr lokalisiert. Des Weiteren beinhaltet sie das sogenannte Brechzentrum und das Miktionszentrum \textsuperscript{\cite[Kap.~6]{trepel2011neuroanatomie}}.





