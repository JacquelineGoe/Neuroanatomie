Dieses Kapitel behandelt die allgemeine Sensorik, \index{Sensorik! allgemein} die im Gegensatz zu speziellen Sensorik (Kapitel~ \ref{sec:spezsens}) steht. In der allgemeinen Sensorik sind die Sinne zusammen gefasst, welche über den ganzen Körper verteilt sind. Dazu gehören unter anderem die Somatosensorik\index{Sensorik! Somato-}, die Propriozeption\index{Propriozeption} und die Viszerosensorik \index{Sensorik! Viszero-} \textsuperscript{\cite[Kap.~22]{kandel2013principles}}. Die spezielle Sensorik \index{Sensorik! speziell} fasst jene Sinne zusammen, welche auf Grund der Cephalisation bei Säugern nach vorne in den Kopf verlagert sind.
\\
\noindent
Die Somatosensorik zeichnet sich durch die Repräsentation der direkten äußeren Welt und der inneren Welt aus. Bei der Repräsentation der Außenwelt unterscheidet man bei Säugetieren zwischen zwei Rezeptorsystemen: Zum einen die haarlose Haut in den Handinnenflächen, an der Fußunterseite, an den Lippen und der Nase, zum anderen die behaarte Haut mit hoch spezialisierten Tastsinneszellen, innerviert durch die Bewegungen des Follikels auf den Blutsinus \index{Blutsinus} der Vibrissen \index{Sinushaar} \textsuperscript{\cite[Kap.~24]{paxinos2014rat}}.
Die Propriozeption kodiert die Informationen der relativen Position der Extremitäten und anderer Körperteile im Raum und bildet in den Vorderextremitäten die Grundlage für abstrakte Wahrnehmung von Objektgrößen und Gewicht. Einen weiteren Bereich in der Somatosensorik bildet der Sinn, der Schmerzen und Temperatur verarbeitet \textsuperscript{\cite[Kap.~24]{paxinos2014rat}}.
In den folgenden Kapiteln wird am Beispiel der Ratte näher auf die unterschiedlichen Systeme und deren Unterschiede eingegangen.

