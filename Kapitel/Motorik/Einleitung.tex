Die Steuerung der willkürlichen Bewegung erfolgt durch das Gehirn. Das zentrale motorische System ist dabei hierarchisch organisiert. Diese Hierarchie der Bewegungskontrolle lässt sich grob in drei Ebenen gliedern. Die oberste Ebene setzt sich aus den Assoziationsfeldern des Neocortex und den Basalganglia des Vorderhirns zusammen. Ihre Aufgabe ist es eine geeignete Strategie festzulegen, um mit einer Bewegung zum Ziel zu gelangen. Die mittlere Ebene befasst sich mit der Taktik der Bewegung. Es wird der zeitliche und räumliche Ablauf genau geplant um eine Bewegung präzise ausführen zu können. Hierzu werden der Motorcortex und das Cerebellum gezählt. Auf der untersten Ebene, bestehend aus dem Hirnstamm und dem Rückenmark, geht es um die eigentliche Ausführung der Bewegung. Das bedeutet es geht um die Aktivierung der Motorneurone, die letztendlich zu einer Kontraktion der Muskeln führt. Auf dieser Ebene wird auch die Körperhaltung kontrolliert und gegebenenfalls auch korrigiert \textsuperscript{\cite[Kap.~14]{neurowissenschaften_baer}}. Im folgenden Kapitel wird auf die wichtigsten motorischen Gebiete und Bahnen eingegangen. Die Basalganglia werden allerdings im darauf folgenden Kapitel~\ref{sec:integrative_systeme} der Integrativen Systeme separat behandelt.   